
\chapter{Installation}

\begin{quotation}

In this chapter you can find all the information needed
to install and run ruote on several operative systems.

In short:
\begin{itemize}
 \item Ruby 1.8.6 or later
 \item rubygems
 \item rufus-dollar, rufus-decision, ....
\end{itemize}


\end{quotation} 

\vspace{1.5in}

\section{Operating system requirements}

You can use whatever O.S. you like, provided this has the possibility 
to run \emph{the ruby platform} required by ruote.

For some operative systems, it is possible to find directly a ruby 
interpreter: Windows, Linux, MacOSX.

You may want to choose also other ruby platforms, like JRuby. The latter
choice is expecially suitable for platforms where there is no ruby interpreter, 
bacause it's only requirement is a \emph{Java Runtime Environment}.


\section{Ruby}

You will need the ruby interpreter with version 1.8.6 or later.

\section{Rubygems}

Additional software needed:
\begin{itemize}
\item rubygems\footnote{This is an optional software package, 
intended to extend your ruby platform. To get  
installation info please refer to http://www.rubygems.org }

\end{itemize}

\section{Rufi}

Additional gems ruote is depending on:

\begin{itemize}

\item rufus-lru (gem)\footnote{TODO: add quick reference to gem basic usage}
\item rufus-scheduler (gem)
\item rufus-dollar (gem)
\item rufus-eval (gem)
\item rufus-verbs (gem)


\end{itemize}

The latter rufus-* gems are pieces of older OpenWFEru project, brought out of ruote because
they are needed for ruote to work, but aren't specific of a BPMS as ruote is. Some of them 
have been also used to leverage other projects(the RESTful scheduling server Taskr is based 
on rufus-scheduler) 

Some parts of ruote are still included in tit but perhaps in the future will be extracted 
in a similar way ( extras/* ).


\section{Ruote}

Now you can install the gem ruote\footnote{The gem ruby package 
manager would have simplificated you job taking advantage of the 
dependencies system, installing directly the ruote gem, gem would
have asked you if you wanted to automatically install required gems}.



