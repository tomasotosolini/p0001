
\chapter{Basic usage}

\section{A simple example}

Now you completed the installation of route. Now we describe a first basic
example of usage of the engine, showing the some of the keys concepts of
everyday ruote life: processes and participants. 

    \subsection{Process description}
        The idea of this process is to ask to a set of two(or possibly more) users what 
        are their impressions about a famous and widely used language, Ruby, in order to 
        make a market study for some further application by some survey agency.

        We choose to make this survey by sending a request mail that contains a small 
        text document with the question(s) to all the selected users. Once they have
        completed, they will reply back with a mail to the original survey agency. When
        all the participants have replied, the agency will create a report describing all
        the answers and the this report is forwarded to the statistical team.

        Ok now we have described the problem, it's time to dive in the ruby/xml coding\ldots

    \subsection{Preparation}

        Being installed as a gem, to use it in your software, you have to write:

        \begin{quote}
        \small{(assuming that your software is saved in \$DIR/survey.rb)}
        \end{quote}

        \begin{quote}
        \begin{verbatim}
        
require 'rubygems' 
    # this is not needed here, but to enable the following 'gem' command
    # you must provide it somewhere in your code (see rubygems for details)

gem 'ruote'
        
        \end{verbatim}
        \end{quote}

        You must have an instance of the engine, so write:
        
        
        \begin{quote}
        \begin{verbatim}
        
require 'openwfe/engine/engine'

#
# instantiating an engine

engine = OpenWFE::Engine.new
        
        \end{verbatim}
        \end{quote}


        Now the engine is instantiated. To enable it to make some more interesting work
        we must now add some participants, i.e. some subjects that perform one ore
        more activities along the process.
        
        \begin{quote}
        \begin{verbatim}
        
#
# adding some participants

engine.register_participant :matz do |workitem|
    puts "matz got the questionary..."
    workitem.matz_comment = ... 
        # you can imagine the content of this string :)
end

engine.register_participant :john do |workitem|
    puts "john  got then the questionary..."
    workitem.john_comment = "Ruby is a great language."
    workitem.john_comment2 = "Java too anyway..."
end

engine.register_participant :linus do |workitem|
    puts "linus got then the questionary..."
    workitem.linus_comment = "I like best C."
    workitem.linus_comment2 = "But last yesterday while waiting 
        for the bus, I wrote down an 
        experimental kernel written in Ruby and it works great."
end

engine.register_participant :summarize do |workitem|
    puts 
    puts "summary of process #{workitem.fei.workflow_instance_id}"
    workitem.attributes.each do |k, v|
        next unless k.match ".*_comment$"
        puts " - #{k} : '#{v}'"
    end
end 
engine.register_participant :summarize_coarse do |workitem|
    puts 
    puts "coarse summary of process #{workitem.fei.workflow_instance_id}"
    workitem.attributes.each do |k, v|
        puts " - #{k} : '#{v}'"
    end
end 
        
        \end{verbatim}
        \end{quote}

        So now we have four participants (it seems that the first three are like humans, 
        and the fourth is more similar to a machine\footnote{The reader should 
        realize that the name "participant`` is choosen to identify an entity that is
        part of the process, and since like in every day of our life there are processes
        that involve both humans and machines here we have such a situation. As example we 
        may say that there are processes involving \emph{only humans}  ( The Tea Taasting 
        Team ...), processes with mixed actors ( ) and processes involving \emph{only machines}
        (Satellite control, Semaphores, Air conditioning control, Medical Equipments)
        
        })

        Now let's define a process to be executed(choosing one that involves the 
        participants already registered). For Ruby lovers:

        \begin{quote}
        \begin{verbatim}

require 'openwfe/def'


class TheProcessDefinition0 < OpenWFE::ProcessDefinition

    sequence do

        concurrence do

            participant :alice
            participant :bob
        end

        participant :summarize
        participant :summarize_coarse
    end
end


        \end{verbatim}
        \end{quote}

        .. or for those loving XML\footnote{Let's assume we saved this XML into a file named \$DIR/process.xml}:

        \begin{quote}
        \begin{verbatim}

<process-definition name="TheProcessDefinition" revision="123">

    <sequence>

        <concurrence>

            <participant ref="alice" />
            <participant ref="bob"
        </concurrence>

        <participant ref="summarize" />
        <participant ref="summarize_coarse" />
    </sequence>
</process-definition >


        \end{verbatim}
        \end{quote}


    \subsection{Execution}

        Now let's see how to use this defintions.

        The first thing to do is start the process. In Ruote this is done feeding it with
        a special object called \textbf{launch item}. Later we shall see that internally 
        the engine operates on objects called \textbf{workitems} and this is only a special
        workitem used to ''bootstrap``.

        \begin{quote}
        \begin{verbatim}

require 'openwfe/workitem'

#
# launching the process

li = OpenWFE::LaunchItem.new(TheProcessDefinition0)

#li = OpenWFE::LaunchItem.new('$DIR/process.xml')
    #
    # XML devotes will use this

li.initial_comment = "please give your impressions
                        about http://ruby-lang.org"

        \end{verbatim}
        \end{quote}

        Ok now we have the initial item, so we can ''lauch`` a new process.

        \begin{quote}
        \begin{verbatim}

fei = engine.launch li
    #
    # 'fei' means FlowExpressionId, the fei returned here is the
    # identifier for the root expression of the newly launched process

puts "started process '#{fei.workflow_instance_id}'"

        \end{verbatim}
        \end{quote}

        And now let's tell the script's main thread to wait until the process thread 
        completed it's life\footnote{Here the reader may discover one of the most 
        important features that come with the software described. Ruote is 
        \emph{completely threaded} ie when ''public`` methods are called the response 
        is not realtime, but instead requests are queued for later execution. }

        \begin{quote}
        \begin{verbatim}

engine.wait_for fei
    #
    # blocks until the process terminates

        \end{verbatim}
        \end{quote}


        So at the end of the work it's time to run the script:

        \begin{quote}
        \begin{verbatim}

$ cd $DIR
(changed dir to $DIR)

$ ruby survey.rb

        \end{verbatim}
        \end{quote}

        and the output should appear as follows:

        \begin{quote}
        \begin{verbatim}

matz got a workitem...
john got a workitem...
linus got a workitem...

summary of process 1173232330251
    - john_comment : 'Ruby is a great language.'
    - matz_comment : ...
    - linus_comment : 'I like best C.'
    - initial_comment : 'please give your impressions 
                            about http://ruby-lang.org'

coarse summary of process 1173232330251
    - linus_comment2 : "But last yesterday while waiting for the 
        bus, I wrote down an 
        experimental kernel written in Ruby and it works great."
    - john_comment : 'Ruby is a great language.'
    - matz_comment : ...
    - john_comment2 : 'Java too anyway...'
    - linus_comment : 'I like best C.'
    - initial_comment : 'please give your impressions 
                            about http://ruby-lang.org'



\end{verbatim}
\end{quote}









\section{A more sophisticated example}

    \subsection{Process description}

    \subsection{Preparation}

    \subsection{Execution}

    \subsection{Analisys}
